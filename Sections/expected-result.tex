\section{Expected Results}
The Boishakh Bengali GPT model is expected to achieve several key outcomes in advancing Bengali natural language processing capabilities:

\subsection{Model Performance}
\begin{itemize}
    \item \textbf{Perplexity Score}: Expected to achieve a perplexity score of 2.5-3.0 on the test dataset, demonstrating strong predictive capabilities for Bengali text generation
    
    \item \textbf{Response Time}: Anticipated average response time of <500ms for generating text sequences up to 256 tokens
    
    \item \textbf{Context Understanding}: The model should maintain coherent context for conversations spanning up to 1024 tokens
\end{itemize}

\subsection{Literary Capabilities}
\begin{itemize}
    \item \textbf{Text Summarization}: 
    \begin{itemize}
        \item Generation of concise summaries for Bengali literary works
        \item Preservation of key themes and narrative elements
        \item Ability to handle both classical and modern Bengali literature
    \end{itemize}
    
    \item \textbf{Literary Analysis}:
    \begin{itemize}
        \item Recognition and analysis of common Bengali literary devices
        \item Understanding of different writing styles and periods
        \item Ability to provide contextual information about authors and their works
    \end{itemize}
\end{itemize}

\subsection{Educational Applications}
\begin{itemize}
    \item \textbf{Academic Support}:
    \begin{itemize}
        \item Assistance with Bengali grammar and syntax
        \item Explanation of complex literary concepts
        \item Generation of practice exercises and examples
    \end{itemize}
    
    \item \textbf{Historical Context}:
    \begin{itemize}
        \item Providing historical background for literary works
        \item Explaining cultural references and traditions
        \item Connecting literary works to their historical periods
    \end{itemize}
\end{itemize}

\subsection{Technical Achievements}
\begin{itemize}
    \item \textbf{Language Processing}:
    \begin{itemize}
        \item Accurate handling of Bengali script and diacritics
        \item Proper processing of compound words
        \item Effective management of dialectal variations
    \end{itemize}
    
    \item \textbf{Search Functionality}:
    \begin{itemize}
        \item Fast and accurate book search capabilities
        \item Context-aware author information retrieval
        \item Efficient navigation of the digital library system
    \end{itemize}
\end{itemize}

\subsection{Model Robustness}
\begin{itemize}
    \item \textbf{Error Handling}:
    \begin{itemize}
        \item Graceful management of spelling variations
        \item Recovery from incomplete or malformed inputs
        \item Appropriate handling of code-mixing (Bengali-English)
    \end{itemize}
    
    \item \textbf{Scalability}:
    \begin{itemize}
        \item Support for concurrent user interactions
        \item Efficient resource utilization
        \item Maintainable system architecture
    \end{itemize}
\end{itemize}

\subsection{User Experience}
\begin{itemize}
    \item \textbf{Interaction Quality}:
    \begin{itemize}
        \item Natural and fluent conversational flow
        \item Contextually appropriate responses
        \item Consistent performance across different user queries
    \end{itemize}
    
    \item \textbf{Accessibility}:
    \begin{itemize}
        \item User-friendly interface for diverse user groups
        \item Support for various input methods
        \item Clear and helpful error messages
    \end{itemize}
\end{itemize}

These expected results align with the project's goal of creating a comprehensive Bengali language model that serves both academic and practical purposes while advancing the state of Bengali natural language processing technology.
\subsection{Normalization}

Normalization is a series of tasks to convert all text on the same standard. And it is done by setting a
single encoding, converting numbers to their word equivalents, and so on. The main goal of
normalization is representing all text in unformat so that all input-output text will be in a standard
format for machine learning problems. It means converting numbers, abbreviations, and special characters, and changed them to clean text following some distinct steps:
\begin{itemize}
    \item \textbf{Set encoding into UTF-8:} All text must be in single encoding format so that all ASCII text should be converted into Unicode. As corpus data is collected from the various sources there is a possibility to have ASCII text. All text must be decoded into a normalized form, such as UTF 8.
    \item \textbf{Other scripts in the corpus:} Bengali text also contains few non-Bengali scripts, especially from English script. It is necessary to transliterate the words from other languages into Bengali.
    \item \textbf{Typos and misspellings:} This is a common phenomenon that text is not always well checked; there could be typos and misspelled words. It is necessary to locating and correcting typos and misspellings. Typo stands for typographical errors like ("receive" \& "receive"),
    \item \textbf{Correcting repeating characters:} Should normalized consciously or unconsciously used repeating contained
    \item \textbf{Popular alternative spelling:} It is necessary to uniform the alternately spelt words into a standard format with a single representation.
    \item \textbf{Expand contractions:} Many types of shortened words mixed with text. It is better if we can revise into a standardized form.
    \item \textbf{Handling numbers:} Text is also mixed different types of numerals like dates, amounts, cardinal, ordinal and nominal etc. The main treatment for numbers is converting numbers into words.
    \item \textbf{Nominal:} Name as a number, phone number, zipcode.
    \item \textbf{Accented characters:} Some diacritics are used to express accented characters like latté, café, etc.
\end{itemize}
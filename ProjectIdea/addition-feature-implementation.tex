\subsection{Additional Feature Implementation in Bengali Text Processing}

\subsubsection{Detecting Spellingn Mistakes and Finding Unusual Words}

The process begins by tokenizing the input Bengali text into individual words using the Indic NLP Library. Each word is then checked against a custom dictionary containing correctly spelled Bengali words. To handle variations in accents, the code normalizes both the words in the text and the dictionary by removing diacritical marks (accents). If a word from the text is not found in the normalized dictionary, it is flagged as misspelled. For each misspelled word, the code suggests possible corrections using fuzzy matching, based on similarity to words in the dictionary. This process enables the detection and correction of spelling mistakes, even for words with accent variations.

\subsubsection{Named Entity Recognition (NER)}

Named Entity Recognition (NER) for Bengali texts involves identifying and classifying specific entities, such as names of people, locations, organizations, and other predefined categories within the text. The process typically begins by tokenizing the Bengali text into words or phrases. Next, a trained model or rule-based system is used to analyze the context of these tokens to determine whether they correspond to known entity types (e.g., person names, geographic locations, etc.). The goal of NER in Bengali is to extract meaningful information from unstructured text, making it useful for applications like information retrieval, question answering, and document categorization.

\subsubsection{N-grams Distribution}

The process begins by tokenizing the Bengali text into individual words or characters. Then, n-grams are generated by grouping consecutive tokens. For example, in a unigram model (1-gram), each word is treated as a single entity, while in a bigram model (2-grams), each pair of consecutive words is grouped together. Similarly, trigram (3-grams) and higher-order n-grams can be created.

These n-grams capture the contextual relationships between adjacent words and are widely used in various natural language processing tasks like language modeling, text classification, and machine translation for Bengali, helping to understand word patterns and improve predictions.